\documentclass[12pt,a4paper]{iopart}

% Document information
\newcommand{\DRN}{CAS-PUB-MDL-000003}
\newcommand{\thisversion}{1.0}

% Author information
\input{common/authors/twhyntie.tex}

% Packages and settings for CERN@school document formatting
\input{common/tools/CERNatschool-packages.tex}

% CERN@school macros and definitions
\input{common/tools/defs.tex}

% Journal short-cuts.
\input{common/bib/journals}

\begin{document}

%
% Title
%
% The CERN@school logo
\input{common/tools/logoheader.tex}

%\title{\titlefont %
\title{%
Maxwell's equations with and without\\
magnetic charge
}

% 
% Author information
\author{\theauthorinit$^{1, \, 2}$}
%
\address{$^1$\theauthoraddressA}
\address{$^2$\theauthoraddressB}
\ead{\mailto{\theauthoremail}}

%-----------------------------------------------------------------------------
% Abstract
\begin{abstract}
Maxwell's equations of classical electromagnetism~\cite{Maxwell1865}
are presented,
in Gaussian units\footnote{
The equations are written in the \emph{Gaussian unit}
system, and not SI units, for simplicity;
in this system the electric field $\electricfield$
and the magnetic field $\magneticfield$ have the same units.},
without (\ref{eq:1},~\ref{eq:2},~\ref{eq:3},~\ref{eq:4})
and
with (\ref{eq:1m},~\ref{eq:2m},~\ref{eq:3m},~\ref{eq:4m})
magnetic charge.
Equations~\ref{eq:1m} to~\ref{eq:4m} are
taken from the front cover of the
Monopole and Exotics Detector at the LHC (MoEDAL)
Technical Design Report~\cite{MoEDAL2009},
the Large Hadron Collider's seventh major experiment
and the latest venture to be undertaken in the search for
Dirac's hypothesised magnetic monopole~\cite{Dirac1931}.
\end{abstract}
%-----------------------------------------------------------------------------

% Add a table of contents.
\setcounter{tocdepth}{1}
\tableofcontents

% Add the license information (CC-BY-4).
\input{common/tools/licenseCCBY4}

\newpage

%%%%%%%%%%%%%%%%%%%%%%%%%%%%%%%%%%%%%%%%%%%%%%%%%%%%%%%%%%%%%%%%%%%%%%%%%%%%%%%
\section{Maxwell's equations without magnetic charge}
%%%%%%%%%%%%%%%%%%%%%%%%%%%%%%%%%%%%%%%%%%%%%%%%%%%%%%%%%%%%%%%%%%%%%%%%%%%%%%%

% Maxwell's equations
\begin{figure}[htbp]
  \centering
\begin{eqnarray}
\label{eq:1} \divvec{\electricfield} & = & 4 \pi \electricchargedensity\\ [10pt]
\label{eq:2} \divvec{\magneticfield} & = & 0\\ [10pt]
\label{eq:3} - \curlvec{\electricfield} & = & \myfrac{1}{\speedoflight} \, \timediff{\magneticfield}\\ [10pt]
\label{eq:4}   \curlvec{\magneticfield} & = & \myfrac{1}{\speedoflight} \, \timediff{\electricfield} + \myfrac{4 \pi}{\speedoflight} \, \electriccurrentdensity
\end{eqnarray}
  \caption[Maxwell's equations of electromagnetism.]
  {\label{fig:maxwellseqs}Maxwell's equations of electromagnetism~\cite{Maxwell1865} with the electric charge and current density terms highlighted (shaded boxes). Note that these are expressed in Gaussian units (not SI) for simplicity.}
\end{figure}

%%%%%%%%%%%%%%%%%%%%%%%%%%%%%%%%%%%%%%%%%%%%%%%%%%%%%%%%%%%%%%%%%%%%%%%%%%%%%%%
\section{Maxwell's equations with magnetic charge}
%%%%%%%%%%%%%%%%%%%%%%%%%%%%%%%%%%%%%%%%%%%%%%%%%%%%%%%%%%%%%%%%%%%%%%%%%%%%%%%

% Maxwell's equations with magnetic charge.
\begin{figure}[htbp]
  \centering
\begin{eqnarray}
\label{eq:1m} \divvec{\electricfield} & = & 4 \pi \electricchargedensity\\ [10pt]
\label{eq:2m} \divvec{\magneticfield} & = & 4 \pi \magneticchargedensity\\ [10pt]
\label{eq:3m} - \curlvec{\electricfield} & = & \myfrac{1}{\speedoflight} \, \timediff{\magneticfield} + \myfrac{4 \pi}{\speedoflight} \, \magneticcurrentdensity\\ [10pt]
\label{eq:4m}   \curlvec{\magneticfield} & = & \myfrac{1}{\speedoflight} \, \timediff{\electricfield} + \myfrac{4 \pi}{\speedoflight} \, \electriccurrentdensity
\end{eqnarray}
  \caption[Maxwell's equations of electromagnetism with magnetic charge.]
  {\label{fig:maxwellseqsmagcharge}Maxwell's equations of electromagnetism with additional terms representing magnetic charge and current density, as shown on the cover of the MoEDAL Technical Design Report~\cite{MoEDAL2009}. Note that these are expressed in Gaussian units (not SI) for simplicity.}
\end{figure}

\clearpage

%
%%%%%%%%%%%%%%%%%%%%%%%%%%%%%%%%%%%%%%%%%%%%%%%%%%%%%%%%%%%%%%%%%%%%%%%%%%%%%%%
% Bibliography
%%%%%%%%%%%%%%%%%%%%%%%%%%%%%%%%%%%%%%%%%%%%%%%%%%%%%%%%%%%%%%%%%%%%%%%%%%%%%%%
%
\section*{References}
\bibliographystyle{unsrt.bst}
\bibliography{MoEDAL}
%
%------------------------------------------------------------------------------

%\newpage

%%%%%%%%%%%%%%%%%%%%%%%%%%%%%%%%%%%%%%%%%%%%%%%%%%%%%%%%%%%%%%%%%%%%%%%%%%%%%%%
\section*{Acknowledgements}
\label{sec:ack}
%%%%%%%%%%%%%%%%%%%%%%%%%%%%%%%%%%%%%%%%%%%%%%%%%%%%%%%%%%%%%%%%%%%%%%%%%%%%%%%
This work was supported by the UK Science and Technology Facilities Council
(STFC) via grant ST/N00101X/1.
%
The author wishes to thank A. Pontzen for many fruitful discussions.


%%%%%%%%%%%%%%%%%%%%%%%%%%%%%%%%%%%%%%%%%%%%%%%%%%%%%%%%%%%%%%%%%%%%%%%%%%%%%%%
\section*{Version History}
%%%%%%%%%%%%%%%%%%%%%%%%%%%%%%%%%%%%%%%%%%%%%%%%%%%%%%%%%%%%%%%%%%%%%%%%%%%%%%%
%______________________________________________________________________________
\begin{table}[h]
\caption{\label{tab:version}Version history.}
\lineup
\begin{indented}
\item[]\begin{tabular}{@{}clc}
\br
\centre{1}{$\quad$Version    $\quad$} & 
\centre{1}{$\quad$Description$\quad$} &
\centre{1}{$\quad$Author     $\quad$} \\
\mr
1.0 & Initial version. & TW \\
\br
\end{tabular}
\end{indented}
\end{table}
%______________________________________________________________________________

\end{document}
