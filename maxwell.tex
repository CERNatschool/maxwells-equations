\documentclass[12pt,a4paper]{iopart}

% Document information
\newcommand{\DRN}{CAS-PUB-MDL-000003}
\newcommand{\thisversion}{1.0}

% Author information
% Author information for T. Whyntie
\newcommand{\theauthorinit}{T Whyntie}
\newcommand{\theauthorfull}{Tom Whyntie}

\newcommand{\theauthoraddressA}{%
Particle Physics Research Centre, %
School of Physics and Astronomy, %
Queen Mary University of London, %
Mile End Road, London, E1 4NS, United Kingdom}
%
\newcommand{\theauthoraddressB}{%
The Institute for Research in Schools, %
Canterbury, %
CT4 7AU, United Kingdom}
%
\newcommand{\theauthoremail}{t.whyntie@qmul.ac.uk}


% Packages and settings for CERN@school document formatting
% Packages and settings for CERN@school document formatting.

% Graphics
\usepackage[xetex]{graphicx}
\usepackage{subcaption}
\usepackage{enumerate}

% Tables
\usepackage{dcolumn}
\newcolumntype{.}{D{.}{.}{2}}

% Layout
\usepackage{layouts}
\usepackage{lscape}
%\usepackage[parfill]{parskip}

% Colours
\usepackage{xcolor}
\definecolor{black}{RGB}{0,0,0}
\definecolor{cernblue}{RGB}{0,85,160}

% For highlighting equation terms with a shaded box.
\newcommand{\highlight}[1]{%
  \colorbox{black!15}{$\displaystyle #1$}}

% Mathematics
\usepackage{latexsym}

% Fonts
\usepackage{fontspec,xunicode}
\defaultfontfeatures{Mapping=tex-text,Scale=MatchLowercase}
\setmainfont{Latin Modern Sans}

% Headers and footers
%
% Using the fancyhdr package for the, um, fancy headings...
\usepackage{fancyhdr}
\setlength{\headheight}{16pt}
\setlength{\voffset}{-0.3in}
\pagestyle{fancy}
%
% Redefine the IOP article header settings for the first page.
\makeatletter
\def\ps@myheadings{%
%    \let\@oddfoot\@empty\let\@evenfoot\@empty
    \let\@oddhead\@empty\let\@evenhead\@empty
    \let\@mkboth\@gobbletwo
    \let\sectionmark\@gobble
    \let\subsectionmark\@gobble}
\makeatother
\lhead{\sffamily\rightmark}
\rhead{\sffamily\leftmark}
\lfoot{\sffamily{CERN@school \DRN-v\thisversion}}
\cfoot{}
\rfoot{\sffamily{\thepage}}

% Exercises/questions
\usepackage[listings]{tcolorbox}
\tcbset{before={\par\medskip\pagebreak[0]\noindent},after={\par\medskip}}%

% Coding
\usepackage{listings}
% Listing options
\definecolor{listbacking}{rgb}{0.98,0.98,0.98}
\definecolor{commentgrey}{rgb}{0.60,0.60,0.60}
\definecolor{keywordblue}{rgb}{0.00,0.00,0.50}
\definecolor{stringgreen}{rgb}{0.00,0.50,0.00}
\definecolor{deepred}{rgb}{0.60,0.00,0.00}
\lstset{
  basicstyle=\footnotesize\ttfamily,
  gobble=4,
  float=hbtp,
  emph={__init__},
  emphstyle=\color{deepred},
  showspaces=false,
  showstringspaces=false,
  backgroundcolor=\color{listbacking},
  commentstyle=\color{commentgrey},
  stringstyle=\color{stringgreen},
  keywordstyle=\color{keywordblue},
  numbers=left,
  numbersep=12pt,
  %framexleftmargin=20pt,
  %framexrightmargin=20pt,
  stepnumber=1,
  showlines=true
}

% Acronyms
\usepackage{acronym}

\usepackage{perpage}
\MakePerPage{footnote}

% Flow charts and diagrams
\usepackage{tikz}
\usetikzlibrary{shapes,shapes.misc,arrows,backgrounds,positioning,fit,calc,shadows}

% Background image.
\usepackage{eso-pic}
\newcommand\AtPageLowerRight[1]{\AtPageLowerLeft{%
   \makebox[\paperwidth][r]{#1}}}
\usepackage{ifthen}

% Links
\usepackage[xetex,pdfpagelabels]{hyperref}%
\hypersetup{%
pdfauthor={\theauthorfull},%
colorlinks=true,%
urlcolor=cernblue,%
citecolor=cernblue,%
%linkcolor=cernblue%
linkcolor=black%
}


% CERN@school macros and definitions
% Numeric prefixes and suffixes
\newcommand{\Giga}{\ensuremath{\textrm{G}}}
\newcommand{\Mega}{\ensuremath{\textrm{M}}}
\newcommand{\kilo}{\ensuremath{\textrm{k}}}

\newcommand{\centi}{\ensuremath{\textrm{c}}}
\newcommand{\milli}{\ensuremath{\textrm{m}}}
\newcommand{\micro}{\ensuremath{\mu}}
\newcommand{\nano}{\ensuremath{\textrm{n}}}

% Units
\newcommand{\unitspec}[1]{\ensuremath{\left[ #1 \right]}}

% Distances
\newcommand{\metre}{\ensuremath{\textrm{m}}}
%
\newcommand{\km}{\ensuremath{\kilo \metre}}
%
\newcommand{\cm}{\centi \metre}
\newcommand{\mm}{\milli \metre}
\newcommand{\um}{\micro \metre}
\newcommand{\nm}{\nano \metre}
%
\newcommand{\detector}{\ensuremath{d}}
\newcommand{\detx}{\ensuremath{x_{\detector}}}
\newcommand{\dety}{\ensuremath{y_{\detector}}}
\newcommand{\detz}{\ensuremath{z_{\detector}}}

% Angles
\newcommand{\degree}{\ensuremath{\,^{\circ}}}
%
% Rotations
\newcommand{\Omegax}{\ensuremath{\Omega_{x}}}
\newcommand{\Omegay}{\ensuremath{\Omega_{y}}}
\newcommand{\Omegaz}{\ensuremath{\Omega_{z}}}
%
% Euler angles
\newcommand{\euleranglea}{\ensuremath{\alpha}}
\newcommand{\eulerangleb}{\ensuremath{\beta}}
\newcommand{\euleranglec}{\ensuremath{\gamma}}
%

% Times
%
\newcommand{\second}{\ensuremath{\textrm{s}}}
\newcommand{\minute}{\ensuremath{\textrm{minute}}}
\newcommand{\minutes}{\ensuremath{\textrm{minutes}}}
\newcommand{\months}{\ensuremath{\textrm{month}}}
%
\newcommand{\us}{\ensuremath{\micro \second}}

\newcommand{\Hertz}{\ensuremath{\textrm{Hz}}}
%
\newcommand{\MHz}{\ensuremath{\Mega \Hertz}}
%
% Experimental quantities (time)
\newcommand{\mytime}{\ensuremath{t}}
\newcommand{\starttime}[1]{\ensuremath{\mytime_{#1}}}
\newcommand{\starttimei}{\starttime{i}}
\newcommand{\acqtime}{\ensuremath{\Delta \mytime}}
\newcommand{\acqtimesub}[1]{\ensuremath{\acqtime_{#1}}}
\newcommand{\acqtimei}{\acqtimesub{i}}

\newcommand{\speedoflight}{\ensuremath{c}}

% Electricity
\newcommand{\volt}{\ensuremath{\textrm{V}}}

% Energy
\newcommand{\energy}{\ensuremath{E}}

\newcommand{\eV}{\ensuremath{\textrm{e\volt}}}

\newcommand{\keV}{\ensuremath{\kilo \eV}}
\newcommand{\MeV}{\ensuremath{\Mega \eV}}

% Charge
\newcommand{\echarge}{\ensuremath{e}}
\newcommand{\chargedensity}{\ensuremath{\rho}}
\newcommand{\currentdensity}{\ensuremath{\vec{j}}}
\newcommand{\electric}{\ensuremath{e}}
\newcommand{\electricfield}{\ensuremath{\vec{E}}}
\newcommand{\electricchargedensity}{\ensuremath{\chargedensity_{\electric}}}
\newcommand{\electriccurrentdensity}{\ensuremath{\currentdensity_{\electric}}}
\newcommand{\magnetic}{\ensuremath{m}}
\newcommand{\magneticfield}{\ensuremath{\vec{B}}}
\newcommand{\magneticchargedensity}{\ensuremath{\chargedensity_{\magnetic}}}
\newcommand{\magneticcurrentdensity}{\ensuremath{\currentdensity_{\magnetic}}}

% Radiation
\newcommand{\Gray}{\ensuremath{\textrm{Gy}}}
\newcommand{\uGy}{\ensuremath{\micro \Gray}}
%
\newcommand{\Sievert}{\ensuremath{\textrm{Sv}}}
\newcommand{\uSv}{\ensuremath{\micro \Sievert}}

% Data
\newcommand{\bits}{\ensuremath{\textrm{b}}}
\newcommand{\bitspersecond}{\ensuremath{\bits \second^{-1}}}

\newcommand{\kbps}{\kilo \bitspersecond}
\newcommand{\Mbps}{\Mega \bitspersecond}

\newcommand{\Bytes}{\ensuremath{\textrm{B}}}

\newcommand{\MBytes}{\ensuremath{\Mega \Bytes}}
\newcommand{\GBytes}{\ensuremath{\Giga \Bytes}}

% Formatting
\newcommand{\term}[1]{{\color{cernblue}\textbf{#1}}}
\newcommand{\bullettext}[1]{\textbf{#1}}
\newcommand{\code}[1]{{ }\textbf{\texttt{#1}}{ }}
\newcommand{\keyp}[1]{\emph{#1}}
\newcommand{\feat}[1]{\emph{#1}}
\newcommand{\bytename}[1]{\textbf{#1}}
\newcommand{\bitsspec}[2]{\texttt{[#1:#2]}}
\newcommand{\menuitem}[1]{\emph{#1}}
%
\newcommand{\radiobutton}[1]{\emph{#1}}
\newcommand{\tickbox}[1]{\emph{#1}}
%
% Windows
\newcommand{\windowstab}[1]{\textbf{#1}}
\newcommand{\windowssection}[1]{\emph{#1}}
\newcommand{\windowsbutton}[1]{\emph{#1}}
% Memory register names.
\newcommand{\memreg}[1]{\texttt{#1}}

% Dates
\newcommand{\thsuper}{\ensuremath{^{\textrm{\scriptsize th}}}}
\newcommand{\ndsuper}{\ensuremath{^{\textrm{\scriptsize nd}}}}
\newcommand{\stsuper}{\ensuremath{^{\textrm{\scriptsize st}}}}

% Timepix
\newcommand{\timepixcounts}{\ensuremath{\texttt{C}}}
\newcommand{\timepixclock}{\ensuremath{T_{c}}}
%
% Detector settings
\newcommand{\THL}{\ensuremath{\textrm{THL}}}
\newcommand{\TEQ}{\ensuremath{\textrm{TEQ}}}
\newcommand{\Ikrum}{\ensuremath{I_{\textrm{\footnotesize{Krum}}}}}

% Calibration
\newcommand{\caliba}{\ensuremath{a}}
\newcommand{\calibb}{\ensuremath{b}}
\newcommand{\calibc}{\ensuremath{c}}
\newcommand{\calibt}{\ensuremath{t}}

% Orientation
\newcommand{\roll}{\ensuremath{\phi_{\textrm{\footnotesize roll}}}}

% Misc. text
%
% CERN@school
\newcommand{\CERNatschool}{CERN@school}
%
% Not applicable
\newcommand{\notapp}{\ensuremath{\textrm{n/a}}}
%
\newcommand{\ith}{\ensuremath{i^{\textrm{\tiny\, th}}}}

% Table tools
\newcommand{\notappc}{\multicolumn{1}{c}{\notapp}}
%\newcommand{\dashc}{\multicolumn{1}{c}{---}}
\newcommand{\dashc}[1]{\multicolumn{#1}{c}{---}}
\newcommand{\dashl}[1]{\multicolumn{#1}{l}{---}}
\newcommand{\vdotsc}{\multicolumn{1}{c}{$\vdots$}}

% Mathematics
\newcommand{\orderof}[1]{\ensuremath{\mathcal{O} \! \left( #1 \right)}}

\newcommand{\probdist}[1]{\ensuremath{\textrm{Pr}(#1)}}
\newcommand{\probdistgiven}[2]{\probdist{#1 \, | \, #2}}

\newcommand{\factorial}[1]{\ensuremath{#1 \, !}}

\newcommand{\expo}[1]{\ensuremath{e^{#1}}}
\newcommand{\fullexpo}[1]{\ensuremath{\textrm{Exp}(#1)}}



\newcommand{\myfrac}[2]{\ensuremath{\frac{#1}{#2}}}

% Calculus

\newcommand{\timediff}[1]{\ensuremath{\myfrac{\partial #1}{\partial t}}}

\renewcommand{\vec}[1]{\ensuremath{\mathbf{#1}}}


% Vector calculus

\newcommand{\nablabold}{\ensuremath{\mathbf{\nabla}}}
\newcommand{\divvec}[1]{\ensuremath{\nablabold \cdot #1}}
\newcommand{\curlvec}[1]{\ensuremath{\nablabold \times #1}}

% Poisson

\newcommand{\mycount}{\ensuremath{n}}
\newcommand{\mycounti}{\ensuremath{\mycount_{i}}}

\newcommand{\poismean}{\ensuremath{\lambda}}


% Journal short-cuts.
% Monthly Notices of the Royal Astronomical Society
\newcommand{\mnras}{MNRAS}


\begin{document}

%
% Title
%
% The CERN@school logo
% To insert the CERN@school logo as a header.
\vspace*{-0.8in}
\hspace*{-0.8in}
\includegraphics[height=1.0in]{common/assets/logos/L000001/doc-header.pdf}



%\title{\titlefont %
\title{%
Maxwell's equations with and without\\
magnetic charge
}

% 
% Author information
\author{\theauthorinit$^{1, \, 2}$}
%
\address{$^1$\theauthoraddressA}
\address{$^2$\theauthoraddressB}
\ead{\mailto{\theauthoremail}}

%-----------------------------------------------------------------------------
% Abstract
\begin{abstract}
Maxwell's equations of classical electromagnetism~\cite{Maxwell1865}
are presented,
in Gaussian units\footnote{
The equations are written in the \emph{Gaussian unit}
system, and not SI units, for simplicity;
in this system the electric field $\electricfield$
and the magnetic field $\magneticfield$ have the same units.},
without (\ref{eq:1},~\ref{eq:2},~\ref{eq:3},~\ref{eq:4})
and
with (\ref{eq:1m},~\ref{eq:2m},~\ref{eq:3m},~\ref{eq:4m})
magnetic charge.
Equations~\ref{eq:1m} to~\ref{eq:4m} are
taken from the front cover of the
Monopole and Exotics Detector at the LHC (MoEDAL)
Technical Design Report~\cite{MoEDAL2009},
the Large Hadron Collider's seventh major experiment
and the latest venture to be undertaken in the search for
Dirac's hypothesised magnetic monopole~\cite{Dirac1931}.
\end{abstract}
%-----------------------------------------------------------------------------

% Add a table of contents.
\setcounter{tocdepth}{1}
\tableofcontents

% Add the license information (CC-BY-4).
%
% The Creative Commons Attribution 4.0 license
%
\begin{figure}[h]
\centering
\includegraphics[height=0.3in]{common/assets/images/CC-BY-4_88x31.png}
\caption*{Except where otherwise noted, this work is licensed under a
\href{http://creativecommons.org/licenses/by/4.0/}{Creative Commons Attribution 4.0 International License}.
Please refer to individual figures, tables, etc. for further information
about licensing and re-use of other content.}
\end{figure}


\newpage

%%%%%%%%%%%%%%%%%%%%%%%%%%%%%%%%%%%%%%%%%%%%%%%%%%%%%%%%%%%%%%%%%%%%%%%%%%%%%%%
\section{Maxwell's equations without magnetic charge}
%%%%%%%%%%%%%%%%%%%%%%%%%%%%%%%%%%%%%%%%%%%%%%%%%%%%%%%%%%%%%%%%%%%%%%%%%%%%%%%

% Maxwell's equations
\begin{figure}[htbp]
  \centering
\begin{eqnarray}
\label{eq:1} \divvec{\electricfield} & = & 4 \pi \electricchargedensity\\ [10pt]
\label{eq:2} \divvec{\magneticfield} & = & 0\\ [10pt]
\label{eq:3} - \curlvec{\electricfield} & = & \myfrac{1}{\speedoflight} \, \timediff{\magneticfield}\\ [10pt]
\label{eq:4}   \curlvec{\magneticfield} & = & \myfrac{1}{\speedoflight} \, \timediff{\electricfield} + \myfrac{4 \pi}{\speedoflight} \, \electriccurrentdensity
\end{eqnarray}
  \caption[Maxwell's equations of electromagnetism.]
  {\label{fig:maxwellseqs}Maxwell's equations of electromagnetism~\cite{Maxwell1865} with the electric charge and current density terms highlighted (shaded boxes). Note that these are expressed in Gaussian units (not SI) for simplicity.}
\end{figure}

%%%%%%%%%%%%%%%%%%%%%%%%%%%%%%%%%%%%%%%%%%%%%%%%%%%%%%%%%%%%%%%%%%%%%%%%%%%%%%%
\section{Maxwell's equations with magnetic charge}
%%%%%%%%%%%%%%%%%%%%%%%%%%%%%%%%%%%%%%%%%%%%%%%%%%%%%%%%%%%%%%%%%%%%%%%%%%%%%%%

% Maxwell's equations with magnetic charge.
\begin{figure}[htbp]
  \centering
\begin{eqnarray}
\label{eq:1m} \divvec{\electricfield} & = & 4 \pi \electricchargedensity\\ [10pt]
\label{eq:2m} \divvec{\magneticfield} & = & 4 \pi \magneticchargedensity\\ [10pt]
\label{eq:3m} - \curlvec{\electricfield} & = & \myfrac{1}{\speedoflight} \, \timediff{\magneticfield} + \myfrac{4 \pi}{\speedoflight} \, \magneticcurrentdensity\\ [10pt]
\label{eq:4m}   \curlvec{\magneticfield} & = & \myfrac{1}{\speedoflight} \, \timediff{\electricfield} + \myfrac{4 \pi}{\speedoflight} \, \electriccurrentdensity
\end{eqnarray}
  \caption[Maxwell's equations of electromagnetism with magnetic charge.]
  {\label{fig:maxwellseqsmagcharge}Maxwell's equations of electromagnetism with additional terms representing magnetic charge and current density, as shown on the cover of the MoEDAL Technical Design Report~\cite{MoEDAL2009}. Note that these are expressed in Gaussian units (not SI) for simplicity.}
\end{figure}

\clearpage

%
%%%%%%%%%%%%%%%%%%%%%%%%%%%%%%%%%%%%%%%%%%%%%%%%%%%%%%%%%%%%%%%%%%%%%%%%%%%%%%%
% Bibliography
%%%%%%%%%%%%%%%%%%%%%%%%%%%%%%%%%%%%%%%%%%%%%%%%%%%%%%%%%%%%%%%%%%%%%%%%%%%%%%%
%
\section*{References}
\bibliographystyle{unsrt.bst}
\bibliography{MoEDAL}
%
%------------------------------------------------------------------------------

%\newpage

%%%%%%%%%%%%%%%%%%%%%%%%%%%%%%%%%%%%%%%%%%%%%%%%%%%%%%%%%%%%%%%%%%%%%%%%%%%%%%%
\section*{Acknowledgements}
\label{sec:ack}
%%%%%%%%%%%%%%%%%%%%%%%%%%%%%%%%%%%%%%%%%%%%%%%%%%%%%%%%%%%%%%%%%%%%%%%%%%%%%%%
This work was supported by the UK Science and Technology Facilities Council
(STFC) via grant ST/N00101X/1.
%
The author wishes to thank A. Pontzen for many fruitful discussions.


%%%%%%%%%%%%%%%%%%%%%%%%%%%%%%%%%%%%%%%%%%%%%%%%%%%%%%%%%%%%%%%%%%%%%%%%%%%%%%%
\section*{Version History}
%%%%%%%%%%%%%%%%%%%%%%%%%%%%%%%%%%%%%%%%%%%%%%%%%%%%%%%%%%%%%%%%%%%%%%%%%%%%%%%
%______________________________________________________________________________
\begin{table}[h]
\caption{\label{tab:version}Version history.}
\lineup
\begin{indented}
\item[]\begin{tabular}{@{}clc}
\br
\centre{1}{$\quad$Version    $\quad$} & 
\centre{1}{$\quad$Description$\quad$} &
\centre{1}{$\quad$Author     $\quad$} \\
\mr
1.0 & Initial version. & TW \\
\br
\end{tabular}
\end{indented}
\end{table}
%______________________________________________________________________________

\end{document}
